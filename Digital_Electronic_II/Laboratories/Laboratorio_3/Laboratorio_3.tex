\documentclass[twocolumn]{IEEEtran}
\usepackage[utf8x]{inputenc}
%\usepackage[activeacute,spanish]{babel}
\usepackage{graphicx}
\usepackage{times}
\usepackage[tbtags]{amsmath}
\usepackage{cite}
\usepackage{slashbox}
\usepackage{pict2e}
\usepackage{float}
\usepackage[all]{xy}
\usepackage{graphics,graphicx,color,colortbl}
\usepackage{subfigure}
\usepackage{wrapfig}
\usepackage{multicol}
\usepackage{colortbl}
\usepackage{url}
\usepackage[tbtags]{amsmath}
\usepackage{amsmath,amssymb,amsfonts,amsbsy}
\usepackage{bm}
\usepackage{algorithm}
\usepackage{algorithmic}
\usepackage[centerlast, small]{caption}
\usepackage[colorlinks=true, citecolor=blue, linkcolor=blue, urlcolor=blue,
breaklinks=true]{hyperref}

\begin{document}
\title{Comparación de aplicaciones en Hardware y Software}
\author{David Ricardo Martínez Hernández Código: $261931$\\
	Juan Sebastian Roncancio Arevalo Código: $261585$}
\maketitle
\markboth{Universidad Nacional de Colombia}{}
\floatname{algorithm}{Algoritmo}
\begin{abstract}
  Se realizo una comparación entre el comando mul, la multiplicación por corrimiento y sumas sucesivas para determinar cual de las 3 instrucciones se ejecuta más rápido. Se utilizaron 3 rutinas diferentes en C, y  se obtuvieron los 3 códigos en lenguaje ensamblador. Falta el que se obtuvo.
\end{abstract}

\begin{keywords}
 Corrimiento, Hardware, LM32, Multiplicación, Procesador, Software, Suma.
\end{keywords}

\section{}

\section{}

\bibliographystyle{ieeetran}
\begin{thebibliography}{99}
\bibitem{harris} Harris, David \& Harris, Sarah.
{\em "`Digital desing and computer architecture"'}.
Pretince Hall, 2003.

\bibitem{patterson} Patterson, David \& Hennessy John
{\em "`Computer Organization And Design - The Hardware-Software Interface"'}.
Kindle Edition, Fourth Edition, 2006.

\end{thebibliography}
\end{document}